\documentclass{tufte-handout}

\usepackage{url}
\usepackage{listings}
\usepackage{graphicx}

\lstset{language=R
  , breaklines = true}

% add numbers to chapters, sections, subsections
\setcounter{secnumdepth}{2}

% chapter format
\titleformat{\chapter}%
            {\huge\rmfamily\itshape\color{red}}% format applied to label+text
            {\llap{\colorbox{red}{\parbox{1.5cm}{\hfill\itshape\huge\color{white}\thechapter}}}}% label
            {2pt}% horizontal separation between label and title body
            {}% before the title body
            []% after the title body

% section format
\titleformat{\section}%
            {\normalfont\Large\itshape\color{orange}}% format applied to label+text
            {\llap{\colorbox{orange}{\parbox{1.5cm}{\hfill\color{white}\thesection}}}}% label
            {1em}% horizontal separation between label and title body
            {}% before the title body
            []% after the title body

\title{Programming your Pictures: R}
\author{Aidan Delaney \& Brent Yorgey}

\begin{document}
\maketitle
\section*{Introduction}
This is a worksheet to support the ``Programming your Pictures'' workshop at Diagrams 2016.  There are two versions of this worksheet, one with answers and one without.  If you'd like the version with the answers then have a look at \sidenote{\url{http://aidandelaney.github.io/handouts/2016DiagramsRTutorial-answers.pdf}}.  In these questions we will use the \lstinline{mpg} dataset shipped with \lstinline{ggplot2}.  You might also find the cheat sheet at \sidenote{\url{https://www.rstudio.com/wp-content/uploads/2015/03/ggplot2-cheatsheet.pdf}} to be helpful.

\section*{First Plot}

\section*{Exercise 1: Plot and colour}
Plot the \lstinline{mpg} of cars for all manufacturers.  An example plot can be seen in figure~\ref{fig:mpg-per-manufacturer}.  Can you add in some colour based on the number of cylinders in the individual model?
\begin{marginfigure}
  \includegraphics[width=\textwidth]{images/mpg-per-manufacturer}
  \caption{Plotting mpg for each manufacturer}
  \label{fig:mpg-per-manufacturer}
\end{marginfigure}

\noindent Furthermore, can the visualisation be coaxed to provide the specific model for the car with the highest \lstinline{mpg}?
\ifanswers
The following will plot car \lstinline{mpg} over the \lstinline{manufacturer}
\begin{lstlisting}
ggplot(mpg, aes(x=factor(manufacturer), y=hwy)) + geom_point(aes(colour=factor(cyl))) + scale_color_brewer(type="div")
\end{lstlisting}

If you want to unveil the name of the best performing model we can change our \lstinline{geom_point} to \lstinline{geom_text}.  It's not very pretty, but it works for data exploration.
\begin{lstlisting}
ggplot(mpg, aes(x=factor(manufacturer), y=hwy)) + geom_text(aes(label=model,colour=factor(cyl))) + scale_color_brewer(type="div")
\end{lstlisting}
\fi

\section*{Exercise 2: Histogram}
Generate a histogram of the \lstinline{mpg} values for cars.  You are advised to examine the tutorial notes for an example and some help\sidenote{\url{http://aidandelaney.github.io/}}.

\ifanswers
The following will plot the histogram of all \lstinline{mpg} values
\begin{lstlisting}
ggplot(mpg, aes(x=hwy)) + geom_histogram()
\end{lstlisting}

It could be more useful if we consider cars to the nearest 5 mpg.
\begin{lstlisting}
ggplot(mpg, aes(x=hwy)) + geom_histogram(binwidth=5)
\end{lstlisting}

Or maybe you'd prefer minor breaks every 2 mpg?
\begin{lstlisting}
ggplot(mpg, aes(x=hwy)) + geom_histogram(binwidth=5) + scale_x_continuous(minor_breaks = c(2:max(mpg$hwy)))
\end{lstlisting}
\fi

\section*{Exercise 3: Fit a linear model}

It might be useful to see the relationship between the cars' \lstinline{city} and \lstinline{hwy} mpg listings.  Please generate a point plot of one over the other and fit a linear model to the points.

\ifanswers
For this we have to plot both the points and the smoothed conditional mean.  We also have to set the right method for what we want to achieve.
\begin{lstlisting}
ggplot(mpg, aes(x=hwy, y=cty)) + geom_point() + geom_smooth(method="lm")
\end{lstlisting}

This plot might look better in a paper if there were less around it, and the line was dashed.
\begin{lstlisting}
ggplot(mpg, aes(x=hwy, y=cty)) + geom_smooth(method="lm", linetype="dashed") + theme_minimal()
\end{lstlisting}
\fi

\bibliographystyle{plain}
\bibliography{bibliography}

\end{document}
