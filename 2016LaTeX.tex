\documentclass[ignorenonframetext,]{beamer}
\setbeamertemplate{caption}[numbered]
\setbeamertemplate{caption label separator}{:}
\setbeamercolor{caption name}{fg=normal text.fg}
\usepackage{amssymb,amsmath}
\usepackage{ifxetex,ifluatex}
\usepackage{fixltx2e} % provides \textsubscript
\usepackage{lmodern}
\ifxetex
  \usepackage{fontspec,xltxtra,xunicode}
  \defaultfontfeatures{Mapping=tex-text,Scale=MatchLowercase}
  \newcommand{\euro}{€}
\else
  \ifluatex
    \usepackage{fontspec}
    \defaultfontfeatures{Mapping=tex-text,Scale=MatchLowercase}
    \newcommand{\euro}{€}
  \else
    \usepackage[T1]{fontenc}
    \usepackage[utf8]{inputenc}
      \fi
\fi
% use upquote if available, for straight quotes in verbatim environments
\IfFileExists{upquote.sty}{\usepackage{upquote}}{}
% use microtype if available
\IfFileExists{microtype.sty}{\usepackage{microtype}}{}
\usepackage{color}
\usepackage{fancyvrb}
\newcommand{\VerbBar}{|}
\newcommand{\VERB}{\Verb[commandchars=\\\{\}]}
\DefineVerbatimEnvironment{Highlighting}{Verbatim}{commandchars=\\\{\}}
% Add ',fontsize=\small' for more characters per line
\newenvironment{Shaded}{}{}
\newcommand{\KeywordTok}[1]{\textcolor[rgb]{0.00,0.44,0.13}{\textbf{{#1}}}}
\newcommand{\DataTypeTok}[1]{\textcolor[rgb]{0.56,0.13,0.00}{{#1}}}
\newcommand{\DecValTok}[1]{\textcolor[rgb]{0.25,0.63,0.44}{{#1}}}
\newcommand{\BaseNTok}[1]{\textcolor[rgb]{0.25,0.63,0.44}{{#1}}}
\newcommand{\FloatTok}[1]{\textcolor[rgb]{0.25,0.63,0.44}{{#1}}}
\newcommand{\CharTok}[1]{\textcolor[rgb]{0.25,0.44,0.63}{{#1}}}
\newcommand{\StringTok}[1]{\textcolor[rgb]{0.25,0.44,0.63}{{#1}}}
\newcommand{\CommentTok}[1]{\textcolor[rgb]{0.38,0.63,0.69}{\textit{{#1}}}}
\newcommand{\OtherTok}[1]{\textcolor[rgb]{0.00,0.44,0.13}{{#1}}}
\newcommand{\AlertTok}[1]{\textcolor[rgb]{1.00,0.00,0.00}{\textbf{{#1}}}}
\newcommand{\FunctionTok}[1]{\textcolor[rgb]{0.02,0.16,0.49}{{#1}}}
\newcommand{\RegionMarkerTok}[1]{{#1}}
\newcommand{\ErrorTok}[1]{\textcolor[rgb]{1.00,0.00,0.00}{\textbf{{#1}}}}
\newcommand{\NormalTok}[1]{{#1}}
\usepackage{graphicx}
\makeatletter
\def\maxwidth{\ifdim\Gin@nat@width>\linewidth\linewidth\else\Gin@nat@width\fi}
\def\maxheight{\ifdim\Gin@nat@height>\textheight0.8\textheight\else\Gin@nat@height\fi}
\makeatother
% Scale images if necessary, so that they will not overflow the page
% margins by default, and it is still possible to overwrite the defaults
% using explicit options in \includegraphics[width, height, ...]{}
\setkeys{Gin}{width=\maxwidth,height=\maxheight,keepaspectratio}

% Comment these out if you don't want a slide with just the
% part/section/subsection/subsubsection title:
\AtBeginPart{
  \let\insertpartnumber\relax
  \let\partname\relax
  \frame{\partpage}
}
\AtBeginSection{
  \let\insertsectionnumber\relax
  \let\sectionname\relax
  \frame{\sectionpage}
}
\AtBeginSubsection{
  \let\insertsubsectionnumber\relax
  \let\subsectionname\relax
  \frame{\subsectionpage}
}

\setlength{\parindent}{0pt}
\setlength{\parskip}{6pt plus 2pt minus 1pt}
\setlength{\emergencystretch}{3em}  % prevent overfull lines
\setcounter{secnumdepth}{0}

\title{Writing Documents with \(\LaTeX\)}
\author{Aidan Delaney}
\date{aidan@ontologyengineering.org \textbar{} @aidandelaney}

\begin{document}
\frame{\titlepage}

\begin{frame}{About Me}

\begin{itemize}
\itemsep1pt\parskip0pt\parsep0pt
\item
  Academic for over a decade (PhD in CS involving logic).
\item
  Researcher in Visual Languages and Visual Reasoning.
\item
  Shipped code in Haskell, C, Java, Perl, Python, C++, JavaScript \&
  others.
\item
  Director of an Eastbourne not-for-profit
  \href{http://techresort.co.uk/}{\textbf{TechResort}}.
\item
  Founder of a submarine-mode startup.
\end{itemize}

\end{frame}

\begin{frame}{Document Preparation}

\begin{itemize}
\itemsep1pt\parskip0pt\parsep0pt
\item
  I'm good with source code.
\item
  I \emph{do} source control.
\item
  I require good support for math formulae.
\item
  I need source code snippets.
\item
  I need \emph{excellent} bibliography support.
\item
  Cross-reference support is necessary.
\end{itemize}

\end{frame}

\begin{frame}[fragile]{Source Code}

\begin{verbatim}
\documentclass{article}

\title{Something}
\author{Aidan}

\begin{document}
My first document.
\end{document}
\end{verbatim}

\end{frame}

\begin{frame}[fragile]{Source Control}

\begin{Shaded}
\begin{Highlighting}[]
\KeywordTok{commit} \NormalTok{e71a29bef2cf2a4f9e3584c297c99285bcf16435}
\KeywordTok{Author}\NormalTok{: Aidan Delaney }\KeywordTok{<}\NormalTok{aidan@ontologyengineering.org}\KeywordTok{>}
\KeywordTok{Date}\NormalTok{:   Thu Jan 21 10:08:08 2016 +0000}

    \KeywordTok{Enabled} \NormalTok{beamer output with automagic conversion of SVG to PDF/PNG etc...}

\KeywordTok{commit} \NormalTok{cef2fe4fdf21145d3565d816fba0eb1389532990}
\KeywordTok{Author}\NormalTok{: Aidan Delaney }\KeywordTok{<}\NormalTok{aidan@ontologyengineering.org}\KeywordTok{>}
\KeywordTok{Date}\NormalTok{:   Mon Jan 18 09:59:57 2016 +0000}

    \KeywordTok{Added} \NormalTok{KTP presentation to index.}

\KeywordTok{commit} \NormalTok{51a3c731d492622fe2c44eb037ece6de57e3bbce}
\KeywordTok{Author}\NormalTok{: Aidan Delaney }\KeywordTok{<}\NormalTok{aidan@ontologyengineering.org}\KeywordTok{>}
\KeywordTok{Date}\NormalTok{:   Mon Jan 18 09:56:37 2016 +0000}

    \KeywordTok{Reveal.js} \NormalTok{bump}
\end{Highlighting}
\end{Shaded}

\end{frame}

\begin{frame}{Mathematics Support}

I might need inline \(e^{\pi i}=-1\) equations.

Or something in a block: \[
\bigcup\limits_{z\in MZ(d)}\Psi(z) = \emptyset
\]

Or even something much more complicated: \[
f^\prime(s^\prime)=\left\{
\begin{array}{cr}
(k,(in\cup\{l\},out))&\textrm{where }\eta(\pi^{-1}(s^\prime))\in\Psi(in\cup\{l\},out)\\
(k,(in,out\cup\{l\}))&\textrm{where }\eta(\pi^{-1}(s^\prime))\in\Psi(in,out\cup\{l\})
\end{array}
\right.
\]

\end{frame}

\begin{frame}[fragile]{Source Code Snippits}

\begin{Shaded}
\begin{Highlighting}[]
\CommentTok{/**}
\CommentTok{ * Calculates the points of intersection.}
\CommentTok{ */}
\KeywordTok{protected} \NormalTok{Collection<Point2D>}
           \FunctionTok{intersectionPoints}\NormalTok{(SplitArcBoundary other) \{}
  \NormalTok{Collection<Point2D> ixs = }\KeywordTok{new} \NormalTok{HashSet<Point2D>();}
  \KeywordTok{for}\NormalTok{(CircleArc2D a1 : }\KeywordTok{this}\NormalTok{.}\FunctionTok{curves}\NormalTok{) \{}
    \KeywordTok{for}\NormalTok{(CircleArc2D a2 : other.}\FunctionTok{curves}\NormalTok{) \{}
      \NormalTok{Optional<Collection<Point2D>> is =}
              \FunctionTok{nonTangentalIntersections}\NormalTok{(a1, a2);}
      \KeywordTok{if}\NormalTok{(is.}\FunctionTok{isPresent}\NormalTok{()) \{}
        \NormalTok{ixs.}\FunctionTok{addAll}\NormalTok{(is.}\FunctionTok{get}\NormalTok{());}
      \NormalTok{\}}
    \NormalTok{\}}
  \NormalTok{\}}
  \KeywordTok{return} \NormalTok{ixs;}
\NormalTok{\}}
\end{Highlighting}
\end{Shaded}

\end{frame}

\begin{frame}{Bibliography Support}

I need to cite papers from a database (Croucher et al. 2014).

\end{frame}

\begin{frame}{Cross-referencing}

\begin{figure}[h!]
  \includegraphics{images/image.png}
  \caption{Example Image}
  \label{eg:image}
\end{figure}

Cross reference to figure~\ref{eg:image}.

\end{frame}

\begin{frame}{Evaluate LibreOffice and Google Docs}

\begin{itemize}
\itemsep1pt\parskip0pt\parsep0pt
\item
  I'm good with source code -- neither are source code.
\item
  I \emph{do} source control -- neither support my standard workflow.
\item
  I require good support for math formulae -- it's \emph{passable} in
  both, but not great.
\item
  I need source code snippets -- uugh!
\item
  I need \emph{excellent} bibliography support -- both are too complex.
\item
  Cross-reference support is necessary -- YMMV.
\end{itemize}

\end{frame}

\begin{frame}[fragile]{Writing your first LaTeX file}

\begin{Shaded}
\begin{Highlighting}[]
\NormalTok{\textbackslash{}documentclass\{article\}}

\NormalTok{\textbackslash{}title\{Something\}}
\NormalTok{\textbackslash{}author\{Aidan\}}

\NormalTok{\textbackslash{}begin\{document\}}
\NormalTok{\textbackslash{}maketitle}
\NormalTok{My first document.}
\NormalTok{\textbackslash{}end\{document\}}
\end{Highlighting}
\end{Shaded}

\end{frame}

\begin{frame}[fragile]{Compiling your LaTeX document}

There are a number of differing \LaTeX compilers. Just use
\texttt{pdflatex}.

\begin{Shaded}
\begin{Highlighting}[]
\NormalTok{$ }\KeywordTok{pdflatex} \NormalTok{myfile.tex}
\end{Highlighting}
\end{Shaded}

\end{frame}

\begin{frame}[fragile]{Citing work}

\begin{Shaded}
\begin{Highlighting}[]
\NormalTok{\textbackslash{}documentclass\{article\}}

\NormalTok{\textbackslash{}title\{Something\}}
\NormalTok{\textbackslash{}author\{Aidan\}}

\NormalTok{\textbackslash{}begin\{document\}}
\NormalTok{\textbackslash{}maketitle}
\NormalTok{My first document cites some work \textbackslash{}cite\{mykey\}}

\NormalTok{\textbackslash{}bibliographystyle\{plain\}}
\NormalTok{\textbackslash{}bibibliography\{mybibfile\}}

\NormalTok{\textbackslash{}end\{document\}}
\end{Highlighting}
\end{Shaded}

\end{frame}

\begin{frame}[fragile]{Adding lists}

We have three types of list \texttt{enumerate}, \texttt{itemize} and
\texttt{description}:

\begin{Shaded}
\begin{Highlighting}[]
\NormalTok{\textbackslash{}begin\{itemize\}}
  \NormalTok{\textbackslash{}item A list item}
  \NormalTok{\textbackslash{}item Another list item}
\NormalTok{\textbackslash{}end\{itemize\}}
\end{Highlighting}
\end{Shaded}

The \texttt{begin} and \texttt{end} blocks mark the start and end of an
\emph{environment}.

\end{frame}

\begin{frame}[fragile]{Adding sections}

\begin{Shaded}
\begin{Highlighting}[]
\NormalTok{\textbackslash{}section\{Section Title\}}
\end{Highlighting}
\end{Shaded}

Note it's not a \texttt{begin} \& \texttt{end} environment.

\end{frame}

\begin{frame}[fragile]{Adding images}

People argue far too much with images in LaTeX. Here's the skinny -- Use
Inkscape to draw any images you require as SVG. Alternatively download
images as PNG or JPEG.

\begin{Shaded}
\begin{Highlighting}[]
\NormalTok{\textbackslash{}begin\{figure\}}
  \NormalTok{\textbackslash{}begin\{center\} }\CommentTok{% Note Yanqui spelling}
    \NormalTok{\textbackslash{}includegraphics\{images/foo\} }\CommentTok{% no extension}
    \NormalTok{\textbackslash{}caption[width=0.8\textbackslash{}textwidth]\{A figure caption\}}
    \NormalTok{\textbackslash{}label\{fig:foo\}}
  \NormalTok{\textbackslash{}end\{center\}}
\NormalTok{\textbackslash{}end\{figure\}}
\end{Highlighting}
\end{Shaded}

But first you need to use the \texttt{graphicx} package.

\end{frame}

\begin{frame}[fragile]{Adding tables}

Tables are a PITA (Note again the use of \texttt{figure},
\texttt{caption} and \texttt{label}.

\begin{Shaded}
\begin{Highlighting}[]
\NormalTok{\textbackslash{}begin\{figure\}}
  \NormalTok{\textbackslash{}begin\{center\}}
    \NormalTok{\textbackslash{}begin\{tabular\}\{l|l|l|l|l|l|l|l|\}}
    \NormalTok{&a&b&aa&ab&ba&aba&aab\textbackslash{}\textbackslash{}\textbackslash{}hline}
    \NormalTok{a&aa&ab&aa&aab&aba&aab&aab\textbackslash{}\textbackslash{}}
    \NormalTok{b&ba&b&aab&ba&ba&aab&aab\textbackslash{}\textbackslash{}}
    \NormalTok{\textbackslash{}end\{tabular\}}
    \NormalTok{\textbackslash{}caption\{A table of stuff\}}
    \NormalTok{\textbackslash{}label\{tab:stuff\}}
  \NormalTok{\textbackslash{}end\{center\}}
\NormalTok{\textbackslash{}end\{figure\}}
\end{Highlighting}
\end{Shaded}

\end{frame}

\begin{frame}{Cross References}

Any \texttt{\textbackslash{}label\{x\}} can be referenced with a
\texttt{\textbackslash{}ref\{x\}}.

\end{frame}

\begin{frame}{Working \textbf{with} \(\LaTeX\)}

\begin{itemize}
\itemsep1pt\parskip0pt\parsep0pt
\item
  If you're fighting \(\LaTeX\) then you're probably wrong, work
  \emph{with} it, not against it.
\item
  Don't hard position figures until the document is completely finished.
\item
  Don't try and save space (eg: respect a page limit) until the document
  is completely finished.
\item
  Remember that your content is \emph{always} more important than the
  tools you use to develop it.
\end{itemize}

\end{frame}

\begin{frame}{The Twist}

\begin{itemize}
\itemsep1pt\parskip0pt\parsep0pt
\item
  LaTeX is often used in source code comments.
\item
  This presentation was not written in \LaTeX :)
  \begin{itemize}
  \item
    the magic of pandoc, or
  \item
    the power of markup.
  \end{itemize}
\end{itemize}

\end{frame}

\begin{frame}{Thank You}

\begin{itemize}
\itemsep1pt\parskip0pt\parsep0pt
\item
  To you, for giving up your time.
\end{itemize}
\end{frame}

\hyperdef{}{references}{}
\frame{References}\label{references}
\addcontentsline{toc}{section}{References}

\hyperdef{}{ref-croucher:rpaolsorbwgsug}{}
Croucher, Nicholas, Andrew Page, Thomas Connor, Aidan Delaney,
Jacqueline Keane, Stephen Bentley, Julian Parkhill, and Simon Harris.
2014. ``Rapid Phylogenetic Analysis of Large Samples of Recombinant
Bacterial Whole Genome Sequences Using Gubbins.'' \emph{Nucleic Acids
Research}, 1--13.

\end{frame}

\end{document}
